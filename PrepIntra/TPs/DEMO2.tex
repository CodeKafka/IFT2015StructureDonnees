\documentclass{report}
\input{preamble.tex}
\usepackage{circuitikz}
\usepackage[scr]{rsfso}

%===============================
\usetikzlibrary{circuits.ee.IEC}
\usetikzlibrary{arrows,shapes.gates.logic.US,shapes.gates.logic.IEC,calc}
\newcommand{\faketarget}{\oplus\!\!\!\!\odot}




\title{\Huge{Architecture des ordinateurs}\\{IFT2015}\\{Démonstration 2}\\{\textbf{Piles}}}
\author{\huge{Franz Girardin}}
\date{\today}
\lstset{inputencoding=utf8/latin1}

            %%%%%%%%%%%%%%%%%  Sect.       %%%%%%%%%%%%%%%%%%%%%%%%%%%%%%%%%%%%%%%%%%%%%%%%%%%%%%%%%







\begin{document}
\maketitle

\pagebreak

\pagebreak
\begin{multicols*}{3}
  

  \footnotesize


  \paragraph{Exercice 1}  
  Supposons qu'une pile S initialement vide ait effectué 
  un   total de 25
  opérations push, 12 opérations top et 10 opérations 
  pop, dont 3 ont renvoyé null pour indiquer une 
  pile vide. Quelle est la taille actuelle de S ?

  \paragraph{} 
  Si les \textbf{3} opérations \texttt{pop()} effectuées
  ont retourné une valeur \texttt{null}, cela indique 
  qu'elles ont été effectuées en premier. Les 
  opération \texttt{top()} n'affectent pas la taille 
  de la pile. Et les \textbf{25} opérations 
  \texttt{push()} et 7 opération \textbf{pop()}
  auraient pu être effectuées dans n'importe quel 
  ordre. La pile contient alors 
  \begin{align*}
        n(\texttt{ push()}  ) - n(\texttt{ pop() }  )
        = 25 - 7 = 18
  \end{align*}

  \paragraph{Exercice 2}


  Supposons qu'une pile S initialement vide ait effectué 
  dans l’ordre les opérations suivantes : 
  1 pop ; 3 push ; 1 len ; 2 pop ; 1 isEmpty ; 5 push ; 
  2 top ; 7 pop ; 2 push ; 3 top. Quelle est la taille
  actuelle de S ?



  \begin{table}[H]
    \caption {}

    \begin{center}
      \renewcommand{\arraystretch}{1.5}
      \fontfamily{flr}\selectfont
      \footnotesize
          \begin{tabular}{|l|l|}
          \arrayrulecolor{blue}\hline
          \rowcolor{lightBlue}
          \textcolor{myb}{Méthode} & 
          \textcolor{myb}{Taille fin op.}  
          \\
          \hline
          \arrayrulecolor{black}
          \texttt{pop()} & 0
          \\
          \hline
          $3 \times \texttt{push(e)}$  & 3 
          \\
          \hline
          \texttt{len()} & 3  
          \\ 
          \hline 
          $2 \times \texttt{pop()}$ & 1  
          \\ 
          \hline 
          \texttt{isEmpty()} & 1  
          \\ 
          \hline 
          $ 5 \times \texttt{push()}$ & 6  
          \\ 
          \hline 
          $2 \times \texttt{top()}$  & 6  
          \\ 
          \hline 
          $7 \times \texttt{pop()}$  & 0  
          \\ 
          \hline 
          $2 \times \texttt{push()}$  & 2  
          \\ 
          \hline 
          $3 \times \texttt{top()}$  & 2  
          \\ 
          \hline 
          $2 \times \texttt{top()}$  & 2 
          \\ 
          \hline 
              \end{tabular}
  \end{center}
  \end{table}

  \paragraph{Exercice 3 }
  Quelles valeurs sont renvoyées lors de la série suivante d'opérations 
  de pile? push(5), push(3), pop(), push(2), push(8), pop(), pop(), 
  push(9), push(1), pop(), push(7), push(6), pop(), pop(), push(4), 
  pop(), pop().


  \paragraph{} 

  \begin{table}[H]
    \caption {}

    \begin{center}
      \renewcommand{\arraystretch}{1.5}
      \fontfamily{flr}\selectfont
      \footnotesize
          \begin{tabular}{|l|l|l|}
          \arrayrulecolor{blue}\hline
          \rowcolor{lightBlue}
          \textcolor{myb}{Méthode} & 
          \textcolor{myb}{Val. retour} & \textcolor{myb}{Contenu pile}    
          \\
          \hline
          \arrayrulecolor{black}
          \texttt{push(5)} & rien & $\{ 5\}$
          \\
          \hline
          \texttt{push(3)} & rien & $\{ 3, 5\}$
          \\
          \hline
          \texttt{pop()} & 3 & $\{ 5 \}$
          \\ 
          \hline 
          \texttt{push(2)} & rien & $\{ 2, 5 \}$
          \\ 
          \hline 
          \texttt{push(8)} & rien & $\{ 8, 2, 5 \}$
          \\
          \hline 
          \texttt{pop()} & 8 & $\{ 2, 5 \}$
          \\ 
          \hline 
          \texttt{pop()} & 2 & $\{ 5 \}$
          \\ 
          \hline 
          \texttt{psuh(9)} & rien & $\{ 9, 5 \}$
         \\ 
          \hline 
          \end{tabular}
  \end{center}
  \end{table}

\end{multicols*}
\end{document}

